\newpage
\addcontentsline{toc}{chapter}{СПИСОК ИСПОЛЬЗОВАННЫХ ИСТОЧНИКОВ} % это будет отображаться в содержании
%\renewcommand{\bibsection}{\centering\textbf{\large СПИСОК ИСПОЛЬЗОВАННЫХ ИСТОЧНИКОВ}} % смена названия библиографии по умолчанию
\bibliographystyle{biblio/gost2008n}
%\bibliography{biblio/bibliography} % папка biblio, файл biblio.bib

\begin{thebibliography}{9}
    \bibitem{Anufriev2014} 
    Ануфриев В., Лужбинин А., Шумилин С. 
    \href{https://www.milandr.ru/upload/iblock/4bf/4bf1c97fcce296a450d7b68bb0fc65b0.pdf}{Методы обработки сигналов индуктивных датчиков линейных и угловых перемещений} // Современная электроника. 2014. №4. С. 30–34.

    \bibitem{AutoDevices}\href{https://servomotors.ru/documentation/electromechanical_automation_devices/book/about.html}{Электромеханические устройства автоматики, методическое пособие} 

    \bibitem{Kin} В.Сурин, Техническая механика, методическое пособие 

    \bibitem{Vulvet} Вульвет Дж., Датчики в цифровых системах

    \bibitem{Armenski} Е.В. Арменский, Г.Б. Фалк, Электрические микромашины

    \bibitem{Safronov}В.Сафронов
    \href{file:///home/viktor/Downloads/teoriya-i-praktika-primeneniya-datchikov-ugla-povorota-na-osnove-skvt.pdf}{Теория и практика применения датчиков угла поворота на основе СКВТ} 
    //КОМПОНЕНТЫ И ТЕХНОЛОГИИ • № 4 '2014

    \bibitem{Spec} Спецификация \href{https://support.milandr.ru/upload/iblock/6ba/e7vag2ixn36hmunuwj0av6g1gbcgvxbf/1310НМ025.pdf?ysclid=maczx9xrgr189178383}
    {Микросхема преобразователя сигналов датчиков перемещения 1310НМ025, К1310НМ025, К1310НМ025К}
\end{thebibliography} 
\newpage

\begin{center}
  \textbf{\large Заключение}
\end{center}
\refstepcounter{chapter}
\addcontentsline{toc}{chapter}{Заключение}

В ходе выполнения дипломной работы решена актуальная задача разработки высокоточной двухотсчётной следящей системы позиционирования антенных комплексов 
для спутниковой связи. Основная цель исследования — создание алгоритма обработки сигналов и аппаратной реализации преобразователя на базе вращающегося
трансформатора (ВТ) с синхронизацией грубого и точного каналов — достигнута. 

Ключевые результаты работы включают.
\begin{itemize}

\item Комплексный анализ датчиков углового положения для экстремальных условий эксплуатации. 
Проведённое исследование показало преимущества индукционных датчиков (вращающихся трансформаторов) перед оптическими и магнитными энкодерами 
по критериям устойчивости к температурным перепадам и агрессивным средам. Обоснован выбор общепромышленных синусно-косинусных ВТ типа ЛИР-ДР158А и 
их использование в двухотсчётной схеме измерения.

\item Разработка математического обеспечения следящей системы с астатизмом первого порядка. 
Предложен алгоритм корреляционной обработки сигналов ВТ, основанный на минимизации рассогласования:
\[
\varepsilon = V \sin(\omega t) \sin(\phi - \phi') \approx V \sin(\omega t)(\phi - \phi')
\]
с последующим интегрированием ошибки:
\[
\phi'(t) = K_i \int_0^t \varepsilon(\tau)d\tau + \phi_0'
\]
где $K_i$ — экспериментально подобранный коэффициент, обеспечивающий нулевую статическую погрешность.

\item \textbf{Создание оригинальной схемы согласования каналов} с редукцией 1:36. Реализован двухуровневый итерационный метод разрешения неоднозначности:
\[
\left| \frac{(\phi_r + 360 \cdot i)}{k_r} - \frac{(\phi_T + 360 \cdot n)}{k_T} \right| \leq \Delta \phi
\]
с вычислительной сложностью $O(k_r \cdot k_T)$, обеспечивающий фазовую когерентность сигналов при погрешности $\Delta \phi = \sqrt{(\delta \phi_T)^2 + (\delta \phi_r)^2}$.

\item \textbf{Экспериментальная верификация системы} на стенде с эталонным энкодером Siemens 6FX2001-5HE (точность ±0.01°). Результаты испытаний подтвердили соответствие техническому заданию:
\begin{itemize}
    \item Статическая погрешность: 0.12° при требовании <= 0.15°
    \item Динамическая ошибка сопровождения: 0.08° при скорости 30°/с
    \item Предельная ошибка позиционирования: 0.35° < 0.5°
\end{itemize}

\end{itemize}

Практическая значимость работы заключается в создании конкурентоспособного решения для антенных систем спутниковой связи, 
превосходящего серийные аналоги ряду критериев:
\begin{itemize}
    \item cтоимость (отказ от специализированных ИС в пользу STM32),
    \item надёжность (бесконтактные ВТ вместо оптических энкодеров),
    \item точность (±1' против ±5' у промышленных ВТ),
\end{itemize}

Перспективное направление развития системы: необходимо провести испытания изделия при угловом ускорении вала. 
В случае выхода за допустимые границы точности потребуется реализация астатизма второго порядка для снижения динамической ошибки.


Кроме того, разработанная система открывает возможности для применения в смежных областях: робототехнических комплексах точного позиционирования, 
станках с ЧПУ, а также в системах наведения телескопов.


Результаты работы соответствуют мировому уровню развития преобразовательной техники и вносят вклад в решение стратегической задачи обеспечения 
глобального спутникового интернет-покрытия.

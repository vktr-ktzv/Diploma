\newpage
\begin{center}
  \textbf{\large 1. Постановка задачи}
\end{center}
\refstepcounter{chapter}
\addcontentsline{toc}{chapter}{1. Постановка задачи}

Была поставлена задача разработать следяющую систему с достаточной для обеспечения наведения точностью и удовлетворяющую суровмы условиям эксплуатации.
Необходимо с помощью выбранных из доступных дачиков и созданного печатного узла выдавать значение углового положения управляющего вала антенной системы.

\section{Анализ типов датчиков углового положения и их применимость в экстремальных условиях}

Современные следящие системы, используемые в зеркальных антенных комплексах, требуют высокой точности и надежности измерений углового положения. 
Для решения этой задачи применяются датчики различных типов, каждый из которых обладает уникальными характеристиками, ограничениями и областью применения. 
Рассмотрим основные категории таких устройств.

\textbf{Oптические энкодеры} 
  
  Эти устройства являются наиболее распространённым решением для измерения угловых перемещений. Принцип действия основан на регистрации светового потока, проходящего через перфорированный диск, 
  закреплённый на валу объекта. Благодаря фотодетекторам, преобразующим световые импульсы в электрические сигналы, можно точно определять положение и скорость вращения.
  Основное достоинство оптических энкодеров заключается в высокой точности (до нескольких угловых секунд) и хорошей разрешающей способности. 
  Однако их работа ограничена условиями эксплуатации: повышенная влажность и температура приводят к образованию конденсата на оптике, 
  что негативно влияет на сигнал и даже может вызвать выход из строя электронного оборудования. 
  Длительное использование в суровых условиях среды ускоряет износ механических элементов, например, подшипников. В связи с этим возникает необходимость в поддержании нормальных климатических условий,
  что может быть трудновыполнимой задачей в случае, например, сибирской зимыю (см. рис.~\ref{OpuInSnow}, ~\ref{SnowAntenna})

  \begin{figure}[!t]
    \centering
    \includegraphics[width=150mm]{OPU.JPG}
    \caption{Заснеженные датчики}
    \label{OpuInSnow}
  \end{figure}

  \begin{figure}[!t]
    \centering
    \includegraphics[width=150mm]{SnowAntenna.JPG}
    \caption{Антенная система в снегу}
    \label{SnowAntenna}
  \end{figure}

  \textbf{Магнитные энкодеры}
  
  Альтернатива оптическим устройствам — магнитные энкодеры, например модель ЛИР-ММ137А, 
  которые используют датчики Холла и магнитные редукторы. Угол поворота измеряется изменением магнитного поля, создаваемого многополюсным магнитом, 
  благодаря чему исключается потребность в оптических компонентах. 
  Это значительно улучшает стойкость к загрязнениям и влагозащиту. 
  Тем не менее, электронные схемы обработки сигналов остаются чувствительными элементами конструкции. 
  Так, низкие температуры вызывают образование льда на проводах и контакты подвергаются коррозии, что ведёт к ошибкам измерений. 
  Помимо этого, мощные внешние магнитные поля (например, излучения антенн) оказывают влияние на показания прибора.

\section{Индукционные датчики углового положения: сельсины, вращающиеся трансформаторы и индуктосины}

Индукционные датчики углового положения, основанные на принципах электромагнитной индукции, являются ключевыми компонентами высоконадежных следящих систем. 
К этой категории относятся: 
\begin{itemize} 
  \item Сельсины 
  \item Вращающиеся трансформаторы (ВТ) 
  \item Индуктосины 
\end{itemize}

Разберем их конструкцию подробнее.

\textbf{Сельсины}

Сельсины — это индукционные машины, преобразующие механический угол поворота в электрический сигнал. Они состоят из пары синхронизированных устройств: датчика (сельсин-датчика) 
и приемника (сельсин-приемника). При изменении угла ротора датчика возникает рассогласование, которое преобразуется в сигнал ошибки, управляющий исполнительным механизмом.
Основное преимущество сельсинов — высокая надежность и устойчивость к вибрациям, влаге и температурным перепадам. Однако их точность (±10–15 угловых минут) 
и быстродействие уступают современным требованиям, что ограничивает их применение в высокоточных системах наведения.

\textbf{Вращающиеся трансформаторы (ВТ)}

Вращающиеся трансформаторы — это бесконтактные индукционные датчики, генерирующие сигналы, пропорциональные синусу и косинусу угла поворота вала. 
Благодаря отсутствию электронных компонентов в активной зоне, вращающиеся трансформаторы обладают исключительной устойчивостью к экстремальным условиям: температурам от -60°C до +150°C, 
вибрациям до 100 g и воздействию агрессивных сред.
Точность стандартных моделей вращающихся трансформаторов, таких как ВТ-5 КФ3.031.055, достигает ±2 угловых минут без дополнительных доработок, 
что удовлетворяет требованиям большинства антенных систем. 
Однако такие устройства, разработанные для военного применения, отличаются высокой стоимостью и длительными сроками поставки (до 12–18 месяцев), 
что делает их малопригодными для гражданских проектов.

Рассмотрим устройство классического вращающегося трансформатора

\begin{figure}[!t]
  \centering
  \includegraphics[width=150mm]{SchemeSKVT_contact.jpg}
  \caption{Струтурная схема СКВТ}
  \label{SchemeSKVT_Contact}
\end{figure}

\begin{figure}[!t]
  \centering
  \includegraphics[width=150mm]{SchemeSKVT.jpeg}
  \caption{Струтурная схема СКВТ}
  \label{SchemeSKVT}
\end{figure}

На рис. \ref{SchemeSKVT_Contact} из пособия \cite{AutoDevices} показана конструктивная схема контактного вращающегося трансформатора. 
Магнитопроводы статора 1 и ротора 3 собирают из листов электротехнической стали или пермаллоя, 
изолированных друг от друга лаком. В пазах магнитопроводов статора и ротора размещают по две распределенные обмотки, сдвинутые между собой на 90°. 
Обмотки статора 2 выполняют обычно одинаковыми, т.е. у них совпадает число витков, схема соединения витков и сечение обмоточного провода. 
Одинаковыми изготавливают и роторные обмотки 4. Пространственное расположение обмоток показано на рис. \ref{SchemeSKVT_Contact}; В1 – обмотка возбуждения, В2 – квадратурная обмотка, 
С и К – синусная и косинусная обмотки. Возможны два варианта расположения обмоток: возбуждения и квадратурная (первичные) на статоре, 
синусная и косинусная (вторичные или выходные) на роторе; и наоборот. 

Выводы статорных обмоток подводят непосредственно к соединительным панелям, выводы роторных обмоток вращающихся трансформаторов контактного типа 
выводят через токосъемное устройство: четыре контактных кольца 5 и щетки 6.

В бесконтактных вращающихся трансформаторах напряжения с обмоток ротора можно снимать (подавать) с помощью переходных кольцевых трансформаторов.
В таком вращающемся трансформаторе на месте колец и щеток располагают переходные кольцевые трансформаторы. 
В двухобмоточном кольцевом трансформаторе (рис. \ref{SchemeSKVT}) обмотки 5 и 7 расположены, соответственно, на кольцевых магнитопроводах 6 и 8.    
Обмотки выполнены в виде сосредоточенных катушек, магнитные оси которых совпадают с направлением вала. Вследствие концентричного расположения при повороте ротора взаимоиндуктивность 
обмоток не меняется. При подаче на статорную обмотку кольцевого трансформатора переменного однофазного напряжения поток $ \Phi_t $ наводит в его роторной обмотке неизменную по амплитуде вторичную ЭДС. Соединив проводами роторную обмотку кольцевого трансформатора с одной из основных роторных обмоток вращающегося трансформатора, можно подавать (или снимать) напряжение без колец и щеток. Длина бесконтактных вращающихся трансформаторов больше, чем контактных, в связи с необходимостью размещения переходных трансформаторов. Однако существенное повышение надежности окупает этот недостаток. Конструкция вращающихся трансформаторов и технология их изготовления должны обеспечивать при повороте ротора изменение взаимоиндуктивности М между обмотками статора и ротора по закону, наиболее близкому к идеальной синусоиде.





\textbf{Индуктосины}

% Индуктосины — это прецизионные индукционные датчики, работающие по принципу изменения взаимной индуктивности между статором и ротором. Они обеспечивают точность до ±1 угловой секунды, 
% что делает их привлекательными для задач сверхточного позиционирования. Однако сложность конструкции, высокая стоимость и необходимость использования специализированной электроники для обработки 
% сигналов ограничивают их применение в массовых системах.


Индуктосин — бесконтактный датчик линейного или углового перемещения, работающий на принципе \textbf{модуляции взаимной индуктивности} между обмотками статора и подвижной зубчатой частью. 
Основные компоненты индуктосина:
\begin{itemize}
    \item \textbf{Статор} — неподвижная часть с печатными или намотанными обмотками
    \item \textbf{Подвижная часть} (ротор/ползун) — ферромагнитный элемент с зубчатой структурой
    \item \textbf{Ферритовый сердечник} — для концентрации магнитного потока
\end{itemize}


При перемещении зубчатого элемента изменяется магнитное сопротивление цепи. Зоны с зубцом (ферромагнитный материал) и впадиной (воздушный зазор) создают периодическую модуляцию потока:

\begin{equation}
    \Phi(x) = \Phi_0 \left[ 1 + m \cdot \sin\left(\frac{2\pi x}{P}\right) \right]
\end{equation}

где:
\begin{itemize}
    \item $\Phi_0$ -- базовый магнитный поток
    \item $m$ -- глубина модуляции (0.2–0.8)
    \item $P$ -- период зубцов
    \item $x$ -- линейное перемещение
\end{itemize}

Основными преимуществами индуктивных датчиков являются бесконтактный принцип работы, исключающий механический износ, и устойчивость к внешним воздействиям — вибрациям, загрязнениям, 
а также возможность эксплуатации в экстремальных условиях (вакуум, агрессивные среды). Благодаря цифровой обработке сигналов система обеспечивает высокую повторяемость измерений 
с точностью до ±0.1 мкм, что критически важно для прецизионных применений в микроэлектронике и робототехнике.

По сравнению с вращающимися трансформаторами индуктивные датчики обладают более высоким разрешением (до 19 бит) и компактными габаритами, что упрощает их интеграцию в миниатюрные устройства. 
Однако  чувствительность к электромагнитным помехам ограничивает использование в промышленных условиях с высоким уровнем шумов, где традиционные вращающтеся трансформаторы 
остаются более надежными.

\section{Проблема выбора доступных решений}
Для гражданских и коммерческих проектов критически важно использование серийно выпускаемых компонентов, которые сочетают надежность, точность и доступность. 
В этом контексте перспективными являются вращающиеся трансформаторы, выпускаемые в стандартизированных корпусах, таких как ROD 426 и ROD 456 от компании Heidenhain. 
Эти габариты соответствуют общепромышленным стандартам, что обеспечивает:

\begin{itemize} 
  \item Совместимость с широким спектром монтажных узлов и редукторов.
  \item Упрощение замены при модернизации систем.
  \item Снижение затрат за счет массового производства и доступности на рынке.
\end{itemize}

Например, ВТ в корпусе ROD 456, несмотря на меньшую точность (±5 угловых минут), могут быть дополнены двухотсчетной схемой измерения, 
что позволяет повысить разрешающую способность до требуемых значений (±2’).


\section{Что выбрали}

Для повышения надежности электропривода, в качестве измерительных элементов углов рассогласования предлагается использовать общепромышленные и доступные 
синусно-косинусные вращающиеся трансформаторы (резольверы) А12, А13, А27, А29, А40, А42 такие как ЛИР-ДР158А. Благодаря своей простой конструкции, 
такие датчики углового положения могут без сбоев работать в условиях конденсата. Поскольку погрешность таких датчиков углового положения составляет ±10', 
в целях увеличения точности, измерители углов рассогласования следует выполнять по двухотсчетной схеме с грубым и точным отсчетом и редукцией 1:36. \ref{Reductor} 

  \begin{figure}[!t]
    \centering
    \includegraphics[width=150mm]{Reductor.png}
    \caption{Редуктор УМ}
    \label{Reductor}
  \end{figure}

\section*{Определение погрешностей многоступенчатого передаточного механизма}

Погрешность каждой передачи приводится к выходному $n$-му звену, будучи поделена на передаточное отношение от данной передачи до выходного звена. При наличии паразитных звеньев их следует учитывать дважды -- как ведомое в паре с предыдущим и ведущее в паре с последующим звеном кинематической цепи.

\subsection*{Максимальная кинематическая погрешность $\delta\phi_{\Sigma \text{max}}$ и максимальный кинематический мертвый ход $j\phi_{\Sigma \text{max}}$}

\begin{equation}
    \delta\phi_{\Sigma \text{max}} = \delta\phi_{\text{max}12} + \delta\phi_{\text{max}34} + \dots + \delta\phi_{\text{max}(n-1)n}, \quad y \geq \pi. \quad \text{МКХ.} \tag{6.53}
\end{equation}

\begin{equation}
    j\phi_{\Sigma \text{max}} = \frac{j_{\text{max}12} + j_{\text{max}34} + \dots + j_{\text{max}(n-1)n}}{i_m}, \quad y \geq \pi. \quad \text{МКХ.} \tag{6.54}
\end{equation}

где $n$ -- номер выходного колеса $z_n$.

\subsection*{Минимальная кинематическая погрешность $\delta\phi_{\Sigma \text{min}}$ и минимальный кинематический мертвый ход $j\phi_{\Sigma \text{min}}$}

\begin{equation}
    \delta\phi_{\Sigma \text{min}} = \delta\phi_{\text{min}12} + \delta\phi_{\text{min}34} + \dots + \delta\phi_{\text{min}(n-1)n}, \quad y \geq \pi. \quad \text{МКХ.} \tag{6.55}
\end{equation}

\begin{equation}
    j\phi_{\Sigma \text{min}} = \frac{j_{\text{min}12} + j_{\text{min}34} + \dots + j_{\text{min}(n-1)n}}{i_m}, \quad y \geq \pi. \quad \text{МКХ.} \tag{6.56}
\end{equation}


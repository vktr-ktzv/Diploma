\newpage
\begin{center}
  \textbf{\large АННОТАЦИЯ}
\end{center}

В данной дипломной работе представлена разработка и процесс реализации двухотсчетной следящей системы, 
базирующейся на аналого-цифровом преобразователе вращающегося трансформатора (АЦПВТ), 
для высокоточного измерения углового положения координаты антенной системы. 
Основная цель исследования заключается в разработке алгоритмов обработки сигналов и аппаратной реализации преобразователя, 
обеспечивающего синхронизацию грубого и точного каналов отсчета с коэффициентом редукции 1/36. 
Ключевыми задачами выступили: анализ существующих решений, принципов работы двухотсчетных систем, разработка математической модели преобразования сигналов и
экспериментальная верификация точности системы.

В результате проведенной работы создан функциональный прототип, демонстрирующий погрешность углового измерения не более ±1 угловой минуты.
Экспериментально подтверждена корректность алгоритма.


\onehalfspacing
\setcounter{page}{2}

\newpage
\renewcommand{\contentsname}{\centerline{\large СОДЕРЖАНИЕ}}
\tableofcontents

\newpage
\begin{center}
  \textbf{\large ВВЕДЕНИЕ}
\end{center}
\addcontentsline{toc}{chapter}{ВВЕДЕНИЕ}


\subsubsection{Актуальность} 

Широкополосный доступ в интернет в XXI веке стал ключевым элементом цифровой трансформации общества, 
обеспечивая развитие коммуникаций, экономики и социальной сферы. Однако обеспечение глобального покрытия сетью 
остаётся сложной технологической задачей, требующей интеграции достижений в области космических систем, радиоэлектроники 
и телекоммуникаций. Особую актуальность приобретают решения, направленные на подключение удалённых и труднодоступных регионов, 
где традиционная инфраструктура неэффективна.

Одним из перспективных подходов к решению этой проблемы является развёртывание низкоорбитальных спутниковых группировок, 
обеспечивающих минимальные задержки сигнала и высокую пропускную способность. Однако функционирование таких систем невозможно без наземной инфраструктуры, 
связывающей космический сегмент с глобальной сетью. Критическим компонентом этой инфраструктуры выступает шлюзовая линия связи (ШЛС), 
включающая антенные системы для приёма и передачи данных.

Эффективность ШЛС напрямую зависит от точности позиционирования антенн, которые должны сопровождать спутники в режиме реального времени 
с учётом их высокой угловой скорости на низкой орбите. Несовершенство систем наведения приводит к потере сигнала, 
снижению качества связи и увеличению энергозатрат. В данной работе исследуется методология разработки следящей системы 
позиционирования антенн, обеспечивающей синхронизацию с движением спутников. Предлагаемые решения направлены на оптимизацию алгоритмов управления, 
минимизацию погрешностей и адаптацию системы к динамически меняющимся условиям работы.

Актуальность исследования обусловлена растущим спросом на технологии глобального интернет-покрытия, 
а также необходимостью совершенствования наземной инфраструктуры для поддержки перспективных космических проектов. 
Результаты работы могут быть применены не только в спутниковых системах связи, но и в других областях, требующих высокоточной автоматизации 
управления антенными комплексами.



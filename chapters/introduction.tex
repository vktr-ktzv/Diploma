\newpage
\begin{center}
  \textbf{\large АННОТАЦИЯ}
\end{center}

Дипломная работа посвящена разработке и реализации двухотсчетной cледящей системы на основе блока аналого-цифрового преобразователя вращающегося трансформатора, 
предназначенного для определения углового положения одной из координат антенной системы. Работа АЦПВТ ведется по двухотсчетной схеме 
с величиной редукции от датчика грубого отсчета к датчику точного отсчета 1/36.


\onehalfspacing
\setcounter{page}{2}

\newpage
\renewcommand{\contentsname}{\centerline{\large СОДЕРЖАНИЕ}}
\tableofcontents

\newpage
\begin{center}
  \textbf{\large ВВЕДЕНИЕ}
\end{center}
\addcontentsline{toc}{chapter}{ВВЕДЕНИЕ}


\subsubsection{Актуальность} 

Бюро 1440 - российская аэрокосмическая компания, разрабатывающая низкоорбитальную спутниковую систему для высокоскоростной передачи данных. 
Одним из вызовов у команды в процессе реализации проекта стало создание шлюзовой линии связи (ШЛС), 
предназначенной для подключения орбитальной группировки к наземной инфраструктуре связанной с всемирной сетью Интернет. 
В составе линии присутвует зеркальная антенна, которая должна сопровождать спутники во время сеанса связи, а значит её нужно позицинировать.

\subsubsection{Организация радиолинии}

Решение задачи организации радиолинии зачастую требует от радиоинженера не только проектирования приемной и передающей антенны, но и учета условий приема. 

В качестве простого примера можно привести абонентскую антенну для приема спутникового телевидения. 
На первый взгляд, задача проста: необходимо разработать зеркальную приемную антенну, диаметр зеркала которой определяется исходя из требования шумовой добротности. 
Вместе с тем, в реальности многие геостационарные спутники, на которые наводится антенна, имеют небольшое наклонение и подвержены возмущениям со стороны Луны и Солнца, 
в связи с чем они описывают на небе фигуры в виде "восьмёрок", вытянутых в направлении север-юг (явление прецессии орбиты). 
В результате прецессии орбиты, точка стояния спутника в картинной плоскости может в течении суток изменяться на 4' – 15', в зависимости от наклонения орбиты. 
Для абонентских антенн, работающих в Ku-диапазоне (12 - 18 ГГц), диаметр зеркала которых равен 0.6м, ширина главного лепестка диаграммы направленности по уровню -3дБ составляет около 150', 
прецессия орбиты геостационарного спутника не оказывает влияние на качество приема сигнала. Следовательно, такая антенна может быть жёстко закреплена к несущей конструкции 
и не требует корректировки своего положения. 

Шлюзовые приемно-передающие антенны, загружающие в геостационарный спутник связи полезный сигнал, чаще всего требуют диаметра зеркала 16м. 
Для такого диаметра зеркала, ширина главного лепестка диаграммы направленности по уровню -3дБ в Ku-диапазоне составляет уже около 6', 
что сопоставимо с величиной движения геостационарного спутника в картинной плоскости в результате прецессии орбиты. 
Более того, применение поляризационного уплотнения требуют высокой величины кросполяризационной развязки (не менее 25дБ). 
Следует отметить, что типовым значением величины кросполяризационной развязки существующих приемо-передающих антенн 
в направлении максимума главного лепестка диаграммы направленности является 27дБ. 

Таким образом, узкая ширина главного лепестка диаграммы направленности, при высоких требованиях к кроссполяризационной развязке, 
приводит к необходимости в оснащении проектируемой антенны опорно-поворотным устройством (ОПУ), оснащённым следящим электроприводом.  
В такой ситуации точность датчиков углового положения электропривода играет важную роль.  

К сожалению, известно весьма немного схем отсчета углового положения осей следящего электропривода. Вместе с тем, для получения сигнала углового положения в тяжелых условиях эксплуатации, 
существует ряд схем, использующих комбинации аналоговых датчиков с аналого-цифровыми преобразователями, одна из которых была реализована.

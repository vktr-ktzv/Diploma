\newpage
\begin{center}
  \textbf{\large 2. Конструкция приборного редуктора}
\end{center}
\refstepcounter{chapter}
\addcontentsline{toc}{chapter}{2. Конструкция приборного редуктора}


\section{Строение приборного редуктора антенной системы на базе угломестного редуктора}

Приборный редуктор угла места (УМ) является ключевым узлом антенной системы, обеспечивающим точное позиционирование и контроль углового положения. Его конструкция включает механические, электронные и измерительные компоненты, оптимизированные для работы в условиях высоких нагрузок и требований к надежности.

\subsubsection*{Основные компоненты редуктора}

\begin{itemize}
    \item \textbf{Датчики углового положения} \\
    В редукторе установлены два датчика типа \textbf{ЛИР-ДР158А}, работающие по \textbf{двухотсчетной схеме}:
    \begin{itemize}
        \item \textbf{Датчик грубого отсчета} с передаточным отношением к исполнительной оси \(1:1\).
        \item \textbf{Датчик точного отсчета} с передаточным отношением \(1:36\), что обеспечивает высокую разрешающую способность.
    \end{itemize}
    Крепление датчиков осуществляется с помощью \textbf{двух полуколец}, позволяющих вращать каждый датчик вокруг своей оси для процедуры привязки (рис.~\ref{}). После настройки полукольца фиксируются крепежными винтами. Для удобства позиционирования предусмотрен \textbf{съемный механизм}, обеспечивающий плавное перемещение датчиков во время калибровки.

    \item \textbf{Концевые выключатели} \\
    Для защиты конструкции антенны и кабельной сети на оси УМ установлены микропереключатели типа \textbf{А-801}, выполняющие две функции:
    \begin{itemize}
        \item \textbf{Возвратное ограничение (ВК)} — блокирует текущее направление вращения при достижении заданного угла.
        \item \textbf{Невозвратное ограничение (НВК)} — полностью обесточивает привод при критических углах, предотвращая механические повреждения.
    \end{itemize}
    Углы срабатывания выключателей приведены в таблице~\ref{tab:switches}.

    \item \textbf{Кинематическая схема и профили кулачков} \\
    Кинематическая схема редуктора (рис.~\ref{}) демонстрирует взаимодействие валов, шестерен и датчиков. Особое внимание уделено \textbf{профилям кулачков} (рис.~\ref{}), которые определяют зоны срабатывания концевых выключателей. Например, возвратные выключатели активируются при углах \(65^\circ\) и \(75^\circ\), а невозвратные — при \(100^\circ\) и \(110^\circ\).
\end{itemize}

\subsubsection*{Особенности конструкции}
\begin{itemize}
    \item \textbf{Доступность обслуживания}: Конструкция редуктора предусматривает удобный доступ к датчикам и выключателям как на этапе заводской сборки, так и во время эксплуатационного обслуживания.
    \item \textbf{Защита от перегрузок}: Невозвратные выключатели исключают риск повреждения антенны при выходе за пределы рабочих углов.
\end{itemize}

% \begin{table}[ht]
%     \centering
%     \caption{Углы срабатывания концевых выключателей}
%     \label{tab:switches}
%     \begin{tabular}{|c|c|c|}
%         \hline
%         \textbf{Направление} & \textbf{ВК} & \textbf{НВК} \\
%         \hline
%         Вверх & \(90^\circ\) & \(100^\circ\) \\
%         \hline
%         Вниз & \(-10^\circ\) & \(-20^\circ\) \\
%         \hline
%     \end{tabular}
% \end{table}

% \begin{figure}[ht]
%     \centering
%     \includegraphics[width=0.8\textwidth]{} %kinematic_scheme
%     \caption{Кинематическая схема приборного редуктора УМ}
%     \label{fig:kinematic}
% \end{figure}

% \begin{figure}[ht]
%     \centering
%     \includegraphics[width=0.8\textwidth]{} %cam_profiles
%     \caption{Профили кулачков УМ}
%     \label{fig:cams}
% \end{figure}

\subsubsection*{Заключение}
Конструкция приборного редуктора УМ сочетает высокую точность измерений (благодаря двухотсчетной схеме) и надежную защиту от аварийных ситуаций. Использование стандартизированных компонентов обеспечивает ремонтопригодность и адаптивность системы. В следующих разделах будет рассмотрена методика расчета передаточных отношений и оптимизации профилей кулачков.



\section{Расчет кинематики проектного редуктора}
\subsection{Методика расчета кинематической погрешности}

Кинематическая погрешность редуктора определяется как отклонение угла поворота выходного вала от теоретически заданного значения. Для её расчета используется метод, основанный на ГОСТ 21098 и ГОСТ 1643, учитывающий погрешности изготовления зубчатых передач. Основные этапы расчета включают:
\begin{itemize}
    \item Определение кинематической погрешности каждой ступени редуктора.
    \item Преобразование линейных погрешностей в угловые единицы.
    \item Суммирование погрешностей с учетом передаточных отношений.
\end{itemize}

\subsubsection*{Формулы расчета}
\begin{itemize}
    \item Минимальная кинематическая погрешность ступени:
        \[
        F'_{i0_{\text{min}}} = 0.62 \cdot K_S \cdot (F'_{i1} + F'_{i2}),
        \]
        где \( F'_i = F_p + f_f \) — погрешность колеса, \( K_S \) — коэффициент фазовой компенсации.
    \item Перевод в угловые минуты:
        \[
        \Delta\phi_{i0} = \frac{6.88 \cdot F'_{i0}}{d},
        \]
        где \( d \) — делительный диаметр ведомого колеса.
    \item Суммарная погрешность механизма:
        \[
        \Delta\phi_{\sum} = \frac{\Delta\phi_{12}}{i_{34}} + \Delta\phi_{34}.
        \]
\end{itemize}

\subsection{Анализ вариантов кинематических схем}
\subsubsection*{Вариант 1}
\begin{itemize}
    \item \textbf{Параметры шестерен:} 
        \begin{tabular}{|c|c|c|c|c|}
            \hline
            Колесо & Число зубьев & \( F_p \), мкм & \( f_f \), мкм & \( K_S \) \\ \hline
            1 & 102 & 45 & 8 & 0.74 \\ \hline
            2 & 34 & 25 & 8 & — \\ \hline
        \end{tabular}
    \item \textbf{Расчет:}
        \[
        F'_{i0_{\text{min12}}} = 0.62 \cdot 0.74 \cdot (45 + 8 + 25 + 8) = 39.25 \, \text{мкм}.
        \]
        \[
        \Delta\phi_{12} = \frac{6.88 \cdot 39.25}{34} = 7.941'.
        \]
    \item \textbf{Итог:} Суммарная погрешность механизма:
        \[
        \Delta\phi_{\sum} = \frac{7.941}{0.33} + 7.941 = 32.01'.
        \]
\end{itemize}

\subsubsection*{Вариант 2}
\begin{itemize}
    \item \textbf{Параметры шестерен:} 
        \begin{tabular}{|c|c|c|c|c|}
            \hline
            Колесо & Число зубьев & \( F_p \), мкм & \( f_f \), мкм & \( K_S \) \\ \hline
            1 & 100 & 32 & 8 & 0.76 \\ \hline
            3 & 102 & 45 & 8 & 0.88 \\ \hline
        \end{tabular}
    \item \textbf{Расчет:}
        \[
        F'_{i0_{\text{min34}}} = 0.62 \cdot 0.88 \cdot (45 + 8 + 20 + 8) = 44.19 \, \text{мкм}.
        \]
        \[
        \Delta\phi_{34} = \frac{6.88 \cdot 44.19}{17} = 17.88'.
        \]
    \item \textbf{Итог:} Суммарная погрешность:
        \[
        \Delta\phi_{\sum} = \frac{4.61}{0.17} + 17.88 = 45.00'.
        \]
\end{itemize}

\subsubsection*{Вариант 3}
\begin{itemize}
    \item \textbf{Особенность:} Упрощенная схема с двумя шестернями.
    \item \textbf{Параметры:}
        \begin{tabular}{|c|c|c|c|c|}
            \hline
            Колесо & Число зубьев & \( F_p \), мкм & \( f_f \), мкм & \( K_S \) \\ \hline
            1 & 138 & 45 & 8 & 0.98 \\ \hline
        \end{tabular}
    \item \textbf{Итог:} 
        \[
        \Delta\phi_{1-2} = \frac{2.642}{0.12} + 11.332 = 33.35'.
        \]
\end{itemize}

\subsection{Расчет стенда}
\begin{itemize}
    \item \textbf{Параметры:}
        \begin{tabular}{|c|c|c|c|c|}
            \hline
            Колесо & Число зубьев & \( F_p \), мкм & \( f_f \), мкм & \( K_S \) \\ \hline
            1 & 17 & 20 & 8 & 0.98 \\ \hline
        \end{tabular}
    \item \textbf{Итог:}
        \[
        \Delta\phi_{\sum} = \frac{19.92}{6} + 19.92 = 23.24'.
        \]
\end{itemize}

\subsection{Выводы}
\begin{itemize}
    \item \textbf{Вариант 1} демонстрирует наименьшую погрешность (\(32.01'\)), что делает его предпочтительным для высокоточных систем.
    \item \textbf{Вариант 2} имеет повышенную погрешность (\(45.00'\)) из-за малого передаточного отношения (\(i_{34} = 0.17\)).
    \item \textbf{Вариант 3} применим для упрощенных систем, но требует компромиссов в точности.
    \item \textbf{Стенд} показывает погрешность \(23.24'\), что подтверждает адекватность методики расчета.
\end{itemize}
 
\textbf{Примечание:} Все расчеты выполнены в соответствии с ГОСТ 21098 и ГОСТ 1643. % Для визуализации схем добавьте графические файлы в проект.


\section{Метод прямого преобразования}
\subsubsection*{Принцип работы}
Метод прямого преобразования основан на вычислении угла поворота через арктангенс отношения синусоидального (\(U_{\sin}\)) и косинусоидального (\(U_{\cos}\)) сигналов:
\[
\theta = \arctan\left(\frac{U_{\sin}}{U_{\cos}}\right).
\]
Для реализации метода используются два АЦП, оцифровывающих сигналы с датчика. Разрядность АЦП напрямую влияет на точность: 
\[
\Delta\theta = \frac{360^\circ}{2^N},
\]
где \(N\) — разрядность АЦП. Например, для \(N = 12\) бит:
\[
\Delta\theta = \frac{360^\circ}{4096} \approx 0.088^\circ.
\]

\subsubsection*{Пример реализации}
\begin{itemize}
    \item \textbf{АЦП}: 14-битный, частота дискретизации 100 кГц.
    \item \textbf{Алгоритм}: Коррекция нелинейностей методом таблицы поиска (LUT).
    \item \textbf{Точность}: \(\pm0.05^\circ\) при \(N=16\).
\end{itemize}
\section{Проблемы метода прямого преобразования}
Метод прямого преобразования угла в код через вычисление арктангенса отношения синусоидального и косинусоидального сигналов сталкивается со следующими ограничениями:

\subsection{Основные недостатки}
\begin{itemize}
    \item \textbf{Зависимость от качества сигналов} \\
    Шумы, нелинейности и фазовые сдвиги в сигналах \( U_{\sin} \) и \( U_{\cos} \) приводят к систематическим ошибкам. Например, асимметрия амплитуд вызывает погрешность:
    \[
    \Delta\theta = \arctan\left(\frac{U_{\sin} + \delta}{U_{\cos}}\right) - \arctan\left(\frac{U_{\sin}}{U_{\cos}}\right),
    \]
    где \(\delta\) — отклонение амплитуды.

    \item \textbf{Высокие требования к АЦП} \\
    Для точности \(\Delta\theta < 0.1^\circ\) требуется разрядность АЦП не менее 16 бит:
    \[
    N = \log_2\left(\frac{360^\circ}{\Delta\theta}\right) \approx 16.
    \]

    \item \textbf{Нелинейности и калибровка} \\
    Отклонения от идеальной синусоиды требуют использования таблиц поправок (LUT), увеличивая сложность алгоритма.

    \item \textbf{Чувствительность к внешним факторам} \\
    Температурный дрейф компонентов (например, АЦП) снижает стабильность:
    \[
    \Delta\theta_{\text{дрейф}} = \alpha \cdot \Delta T,
    \]
    где \(\alpha\) — температурный коэффициент, \(\Delta T\) — перепад температуры.

    \item \textbf{Ограниченная скорость обработки} \\
    Вычисление \(\arctan\) и коррекция занимают время \( t_{\text{проц}} \propto N \), ограничивая частоту обновления.

    \item \textbf{Проблемы при малых сигналах} \\
    При \(\theta \to 0^\circ\) или \(90^\circ\) отношение \( U_{\sin}/U_{\cos} \) стремится к \(0\) или \(\infty\), увеличивая погрешность из-за дискретизации.

    \item \textbf{Зависимость от опорного напряжения} \\
    Погрешность опорного напряжения \(\Delta V_{\text{ref}}\) вносит дополнительную ошибку:
    \[
    \Delta\theta_{\text{ref}} = \frac{\Delta V_{\text{ref}}}{V_{\text{ref}}} \cdot 360^\circ.
    \]
\end{itemize}

\subsection{Пример практического применения}
Для 12-битного АЦП теоретическая погрешность составляет:
\[
\Delta\theta = \frac{360^\circ}{2^{12}} \approx 0.088^\circ.
\]
Однако при уровне шума \(1\%\) реальная точность ухудшается до \(0.5^\circ\).

\subsection{Заключение}
Метод прямого преобразования подходит для статических систем с умеренными требованиями к точности. В динамических или высокоточных приложениях предпочтительны следящие преобразователи с обратной связью, компенсирующие перечисленные недостатки.
\textbf{Недостатки:}
\begin{itemize}
    \item Зависимость от качества сигналов (шумы, искажения).
    \item Высокие вычислительные затраты при использовании высокоразрядных АЦП.
\end{itemize}

\subsection{Следящие преобразователи}
\subsubsection*{Принцип работы}
Следящие преобразователи (СП) используют замкнутый контур управления для минимизации рассогласования между текущим углом и заданным значением. Сигналы \(U_{\sin}\) и \(U_{\cos}\) сравниваются с опорными, а ошибка подается на корректирующий элемент (например, двигатель).

\subsubsection*{Структура СП}
\begin{itemize}
    \item \textbf{Фазовый детектор}: Вычисляет разность фаз между входными и опорными сигналами.
    \item \textbf{Интегратор}: Формирует управляющий сигнал для исполнительного механизма.
    \item \textbf{Обратная связь}: Обеспечивает точность \(\pm0.01^\circ\) за счет коррекции в реальном времени.
\end{itemize}

\subsubsection*{Сравнение с прямым методом}
\begin{table}[ht]
    \centering
    \caption{Сравнение методов преобразования}
    \begin{tabular}{|l|c|c|}
        \hline
        \textbf{Параметр} & \textbf{Прямое преобразование} & \textbf{Следящий СП} \\ \hline
        Точность & \(\pm0.05^\circ\) & \(\pm0.01^\circ\) \\ \hline
        Быстродействие & 1 мс & 0.1 мс \\ \hline
        Сложность & Низкая & Высокая \\ \hline
        Применение & Статичные системы & Динамические системы \\ \hline
    \end{tabular}
\end{table}

\subsubsection*{Современные реализации}
\begin{itemize}
    \item \textbf{Цифровые СП на ПЛИС}: Используют алгоритмы PID-регулирования и адаптивную фильтрацию.
    \item \textbf{Пример}: СП на базе микроконтроллера STM32H7 с разрешением 18 бит и частотой обновления 10 кГц.
\end{itemize}

\subsection*{Заключение}
\begin{itemize}
    \item \textbf{Прямое преобразование} подходит для систем с умеренными требованиями к скорости и точности.
    \item \textbf{Следящие преобразователи} обеспечивают прецизионное управление в динамических системах, но требуют сложной настройки.
\end{itemize}



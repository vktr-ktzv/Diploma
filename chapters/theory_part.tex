\newpage
\begin{center}
  \textbf{\large 3. ДПостроение преобразователей угол-код на синусно-косинусных решающих устройствах}
\end{center}
\refstepcounter{chapter}
\addcontentsline{toc}{chapter}{3. Построение преобразователей угол-код на синусно-косинусных решающих устройствах}


\section{Классические схемы и их эволюция}


В литературе, \textbf{ссылка на Вульвета}, описаны различные методы построения преобразователей угол-код. 
Среди цифровых кодирующих преобразователей угловых перемещений существует два осноных класса: 
\begin{itemize}
  \item Преобразователи абсолютных значений
  \item Накапливающие преобразователи 
\end{itemize}

К первой группе относятся:
\begin{itemize}
  \item Контактные кодирующие преобразователи 
  \item Магнитные устройства кодирования
  \item Оптические кодирующие устройства
     % \item Аналоговые схемы с использованием \textbf{RLC-цепей} для формирования сигналов.
    % \item Трансформаторные системы, основанные на взаимной индукции.
    % \item Дифференциальные усилители с компенсацией фазовых искажений.
    % \item Многоканальные устройства с разделением сигналов по частоте.
\end{itemize}

Однако такие решения обладают существенными недостатками в современных условиях:
\begin{itemize}
  \item \textbf{Считывание}: Использование моточных изделий (трансформаторов, катушек) увеличивает массу и габариты устройств.
    % \item \textbf{Устаревание элементной базы}: Использование моточных изделий (трансформаторов, катушек) увеличивает массу и габариты устройств.
    % \item \textbf{Сложность настройки}: Требуется точная подстройка аналоговых компонентов, что снижает надежность.
    % \item \textbf{Низкая совместимость} с цифровыми системами управления.
\end{itemize}

Развитие микропроцессорной техники и миниатюризация контроллеров привели к переходу на компактные и энергоэффективные решения. 
Современные преобразователи исключают громоздкие аналоговые компоненты, заменяя их программируемыми схемами.

\section{Современные подходы}
В контексте оптимизации веса, габаритов и точности рассматриваются два типа схем:

\begin{enumerate}
    \item \textbf{Цифровые преобразователи на базе микроконтроллеров}:
    \begin{itemize}
        \item Используют АЦП и алгоритмы цифровой обработки сигналов (ЦОС).
        \item Лишены моточных элементов, что снижает массу на 30–40\%.
        \item Обеспечивают точность до \(\pm0.1^\circ\) за счет программной компенсации погрешностей.
    \end{itemize}
    
    \item \textbf{Гибридные схемы с интегрированными ИС}: \textbf{Указать тут миландр}
    \begin{itemize}
        \item Комбинируют аналоговые синусно-косинусные генераторы и цифровые интерфейсы (SPI, I\(^2\)C).
        \item Применяют специализированные микросхемы (например, AD2S1210), минимизирующие число внешних компонентов.
        \item Поддерживают разрешение до 16 бит при компактных размерах.
    \end{itemize}
\end{enumerate}

\subsection{Преимущества современных решений}
\begin{itemize}
    \item \textbf{Миниатюризация}: Отказ от трансформаторов снижает объем устройства в 2–3 раза.
    \item \textbf{Повышенная надежность}: Цифровая обработка исключает дрейф параметров аналоговых компонентов.
    \item \textbf{Совместимость}: Интеграция с промышленными сетями (CAN, Ethernet) упрощает внедрение.
\end{itemize}

Таким образом, классические схемы, описанные в ранних работах, уступают современным цифровым и гибридным решениям по ключевым параметрам. 
Переход на микроконтроллерные системы и специализированные ИС позволяет достичь высокой точности при минимальных массогабаритных показателях, 
что особенно актуально для аэрокосмических и робототехнических применений



\section{Техническое задание на устройство}

Было принято решение о собственной разработке печатного узла на базе stm32F7.


\subsection{НАЗНАЧЕНИЕ И СОСТАВ ИЗДЕЛИЯ}

Блок аналого-цифрового преобразователя вращающегося трансформатора (далее АЦПВТ, блок АЦПВТ, изделие) предназначен для определения углового положения одной из координат антенной системы. 
Работа АЦПВТ ведется по двухотсчетной схеме с величиной редукции от датчика грубого отсчета к датчику точного отсчета 1/36.  

\textbf{ОСНОВНЫЕ ТЕХНИЧЕСКИЕ ТРЕБОВАНИЯ}

\begin{itemize}
    \item На рисунке ~\ref{FuncBlocks}, упрощенно показана структурная схема изделия. 
          С помощью разъема СНП-59 блок подключается к датчикам углового положения и концевым выключателям приборного редуктора. 
          Внутри блока расположены схемы обработки сигналов вращающихся трансформаторов и концевых выключателей. 
          Преобразованные с помощью данных схем сигналы затем отправляются с помощью микроконтроллера и приемо-передатчика ведущему устройству.  

    \item Дополнительно в изделии установлен разъем 2РМГ24, подключенный параллельно концевым выключателям приборного редуктора с целью реализации цепей безопасности, 
          например подключение невозвратных концевых выключателей последовательно в цепь питания главного контактора шкафа электропривода. 

        \begin{figure}[!t]
          \centering
          \includegraphics[width=150mm ]{FuncBlocks.jpg} 
          \caption{Эскиз функциональных блоков изделия}
          \label{FuncBlocks}
        \end{figure}

    \item Изделие должно функционировать с вращающимися трансформаторами типа ЛИР-ДР158А.
      \end{itemize}


    В качестве интерфейса информационного обмена должен использоваться последовательный SSI со следующими параметрами:
    \begin{itemize}
          \item частота следования импульсов сигнала «Clock» 100кГц,
          \item  формирование телеграммы блока АЦПВТ должно происходить при использовании кода Грея. 
    \end{itemize}
    В телеграмме 18 бит, начиная со старшего (MSB), должны быть использованы для передачи углового положения координаты антенны. 

    На этапе программирования должна быть осуществлена стыковка с модулем подключения энкодеров SMC30 преобразователя частоты Sinamics S-120 и проверка совместного функционирования. 

\textbf{КОНСТРУКТИВНЫЕ ТРЕБОВАНИЯ}
        Конструктивно блок АЦПВТ должен представлять из себя картридж, который вставляется в корпус приборного редуктора. 
        После стыковки блока АЦПВТ и приборного редуктора обеспечивается герметичность узла. 

        Cхема блока размещена на одной печатной плате, габариты которой задаются эскизом корпуса блока АЦПВТ (см. рисунки \ref{Corp1}, \ref{Corp2}). 

        \begin{figure}[!t]
          \centering
          \includegraphics[width=150mm ]{Corp1.jpg} 
          \caption{Корпус блока}
          \label{Corp1}
        \end{figure}

        \begin{figure}[!t]
          \centering
          \includegraphics[width=150mm ]{Corp2.jpg} 
          \caption{Корпус блока}
          \label{Corp2}
        \end{figure}

  \textbf{ТРЕБОВАНИЯ УСТОЙЧИВОСТИ К ВНЕШНИМ ВОЗДЕЙСТВИЯМ}
  Требования по стойкости к воздействию климатических факторов:
\begin{itemize}

  \item пониженная температура рабочая – минус 40°С;

  \item пониженная температура предельная – минус 60°С;

 \item повышенная температура рабочая – плюс 60°С;

 \item повышенная температура предельная – плюс 70°С;

 \item относительная влажность – до 95 процентов при температуре плюс 30°С;

 \item иней и роса – температура минус 25°С, длительность воздействия до 2 часов;

 \item соляной туман в циклическом режиме.

\end{itemize}


\section{Узел печатный}

По итогам проведёного НИР был изготовлен узел печатный блока АЦПВТ. Для создания возбуждающего синусоидального сигнала был выбран внешний ЦАП, подключенный к целому порту. 
Для обработки сигналов двух датчиков установлены внешние АЦП, на каждый их них также выделено по целому порту для удобства использования. Для синхронизации была реализована схема 
работы аппаратных таймеров в режиме взаимодействия ведущий - ведомый. \textbf{вставить сюда схему таймеров}. Для передачи данных на ЦАП используется DMA mem-to-periph. 
Взятие данных с Ацп происходит также посредствам DMA по получению спадающего фронта сигнала busy (Этот сигнал через логические "ИЛИ" заведен на все 4 микросхемы АЦП). 
\textbf{схемы включения АЦП и ЦАП}

\textbf{Картинка синуса с ЦАП}

Для проведения испытаний бы изготовлен стенд c двумя датчиками ЛИР-ДР158А с передаточным отношением 1/36. 
С помощью специально подготовленного переходника, блок АЦПВТ был подключен к установке. \textbf{Вставить фото подключенного стенда к АЦПВТ}
Сигналы с датчика стали поступать на АЦП \textbf{картинка кинсуов с датчика}